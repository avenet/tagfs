\documentclass{article}

\usepackage[spanish]{babel}
\usepackage[utf8]{inputenc}
\usepackage{fullpage}
\usepackage{longtable}
\usepackage{url}
\usepackage{babelbib}

\title{
	\LARGE{TagFS} \\
	\Large{Especificación del Sistema}
}

\author{
  	Abel Puentes Luberta \\
  	\small{abelchino@lab.matcom.uh.cu}
  	\and
  	Andy Venet Pompa \\
  	\small{vangelis@lab.matcom.uh.cu}
  	\and
  	Ariel Hernández Amador \\
  	\small{gnuaha7@uh.cu}
  	\and
  	Yasser González Fernández \\
  	\small{yglez@uh.cu} 	
}

\date{}

\begin{document}

\maketitle

\thispagestyle{empty}

\newpage

\setcounter{page}{1}

\section{Introducción}

Actualmente existe la clasificación de recursos a partir de un conjunto de 
etiquetas. En este contexto, los ficheros constituyen un recurso de particular 
importancia. 

Basándonos en la idea anterior nuestro sistema pretende brindar una plataforma
distribuida para manjear ficheros clasificados según etiquetas donde el acceso
a los mismos este dado precisamente por las etiquetas que lo describen, de tal
modo que el usuario pueda trabajar con los archivos como un sistema de ficheros
clásico, pemitiendo buscar, añadir y eliminar los ficheros existentes en el
sistema.

Nuestro sistema esta compuesto por dos tipos de entidades:
\begin{itemize}
  \item \emph{Servidor:} Un servidor de nuestro sistema representa a un lugar
  donde serán almacenados los recursos. Un servidor realiza la búsqueda entre
  los ficheros que contiene, los almacena físicamente y los elimina, según sea
  la operación a realizar. En ningún caso se realiza una comunicación entre
  servidores.
  \item \emph{Cliente:} Un cliente de nuestro sistema representa la forma de
  interacción del usuario con los recursos almacenados. Un cliente es el que se
  encarga de invocar una operación en el servidor y cuando sea necesario
  procesar los resultados que esta produzca. En ningún caso se realiza una
  comunicación entre clientes.
\end{itemize}

Como puede verse, en nuestro sistema toda la comunicación sigue el modelo
cliente-servidor.

Para lograr un sistema multiplataforma, capaz de funcionar independientemente
del sistema operativo o las características de la máquina, decidimos que nuestro
sistema fuera programado completamente en \emph{Python}. Por lo que el único
requisito su funcionamiento es contar con \verb|Python 2.6| o una versión
superior (de la rama 2.x).

\section{Modelo de comunicación}

Para gestionar la comunicación entre las entidades de nuestro sistema,
utilizamos una plataforma de Invocación Remota de Métodos (RMI\footnote{Remote
Method  Invocation})llamada \emph{Pyro}\footnote{http://pyro.sourceforge.net/}
que se ajusta con el modelo de comuicación \emph{ORB}\footnote{Object Request
Broker}) visto durante el curso. Esta plataforma está implementada
completamente en Python y brinda una forma sencilla y transparente de usar la
programación orientada a objetos entre entidades no necesariamente locales de
nuestro sistema distribuido.

Un elemento escencial de las plataformas RMI son el sistema de nombrado que usa
la misma, en nuestro caso en particular \emph{Pyro} no permite tener más de dos
servidores de nombrado en el sistema, lo que hace peligrar la tolerancia a
fallas de nuestro sistema, pues lo convierte en un sistema con un componente
centralizado. 

Para mantener nuestro sistema totalmente distribuido, decidimos
no utilizar el sistema de nombrado de \emph{Pyro}, sino utilizar una
herramienta de \emph{auto-descubrimiento} que nos permite descubrir
automáticamente los servidores de nuestro sistema en la red, además de
notificarnos ante la aparición o desaparición de alguno de estos en la red. Con
este fin utilizamos una implementación en Python de \emph{Zeroconf}.

\section{Comportamiento y tolerancia a la ocurrencia de fallas}

Como expresamos en la sección anterior nuestro sistema es totalmente
distribuido sin ningún elemento centralizado que pueda hacer peligrar la
estabilidad del mismo ante alguna falla. Ante una falla de un servidor durante
una interacción cliente-servidor, el cliente será notificado de la falla
ocurrida.

De igual manera nuestro sistema permite replicar los ficheros en el sistema en
un porciento de los servidores disponibles determinado por el cliente de
manera tal que, de acuerdo con la disponibilidad de recursos en el sistema,
ante la falla de un servidor, los ficheros no dejen de estar disponibles. Es
importante notar que si el sistema tiene un porciento de replicación pequeño
pueden dejar de estar disponibles algunos ficheros ante la falla de un
servidor, pero en cualquier caso ningún cliente obtendrá estos ficheros como
resultados de una búsqueda.

\section{Entidades del sistema}

En nuestro sistema se manejan tres conceptos fundamentales: ficheros, clientes
y servidores. A continuación se exponen las características de cada uno. Las
operaciones relativas al funcionamiento de la plataforma de comunicación se
omiten en este documento, pero pueden ser consultadas en en la
\emph{Documentación del API}\footnote{Este documento se encuentra en
elaboración} de nuestro proyecto.

\subsection{Ficheros}

Para nuestro sistema un fichero está compuesto de los siguientes elementos:

\begin{itemize}
  \item \emph{Tags}: Este elemento contiene las etiquetas que caracterizan un
  fichero en nuestro sistema. Estas etiquetas constituyen la principal forma de
  búsqueda de nuestro sistema.
  \item \emph{Description}: Este elmento contiene un texto libre, que describ
  a al fichero en cuestión. La descripción del fichero solo se tiene en cuenta
  en caso de realizar una búsqueda de texto libre en el sistema.
  \item \emph{Name}: Nombre del fichero en el sistema de ficheros de origen.
  \item \emph{Content}: Content (tratados como binarios) del fichero.
\end{itemize} 

Cada fichero es identificado unívocamente en nuestro sistema mediante su
contenido, por lo que ficheros con el mismo contenido se van a remplazar unos a
otros, quedando en cada momento el último en añadirse. Esta es la forma que
brinda nuestro sistema para actualizar los metadatos de un fichero.

\subsection{Clientes}

Un cliente de nuestro sistema es una entidad que va a acceder al sistema para
comunicarse con los servidores que se encuentran disponibles, realizar alguna
operación en los mismos e interpretar los resultados. Un cliente puede realizar
las operaciones siguientes:

\begin{itemize}
  \item \emph{Put}: Esta operación añade un fichero al sistema con
  un porciento de replicación, que representa en que porciento de los
  servidores disponibles se almacenará el fichero.
  \item \emph{Get}: Esta operación obtiene el contenido de un fichero específico
  de nuestro sistema a partir del identificador unívoco que lo representa.
  \item \emph{Remove}: Esta operación elimina un fichero del sistema a partir
  del identificador unívoco que lo representa.
  \item \emph{List}: Esta operación lista todos los ficheros de nuestro sistema
  que contengan todas las etiquetas dadas.
  \item \emph{Search}: Esta operación busca un texto libre sobre los datos
  de los ficheros almacenados en todos los servidores y devolviendo los
  ficheros que contienen los términos a buscar en alguno de sus metadatos (en
  esta búsqueda se excluye el contenido del fichero). Como valor a buscar puede
  usarse una consulta que contengan los operadores lógicos \verb|OR| y
  \verb|AND| entre otras opciones. Esta búsqueda se realiza de modo
  distribuido.
  \item \emph{Info}: Esta operación obtiene los metadatos de un fichero
  específico almacenado en el sistema.
\end{itemize}

Cualquier aplicación que realice un conjunto de estas operaciones y se
comunique con los servidores de nuestro sistema puede ser considerada como
cliente de nuestro sistema.

Una especificación más profunda se encuentra en la \emph{Documentación del API}
de nuestro proyecto.

\subsection{Servidores}

Un servidor de nuestro sistema es una entidad que almacena, gestiona ficheros
e interactua con los clientes. Un servidor puede realizar las siguientes
operaciones:

\begin{itemize}
  \item \emph{Status}: Esta operación brinda información relativa al estado del
  servidor, tales como la capacidad y el espacio disponible en el mismo.
  \item \emph{Put}: Esta operación almacena físicamente un fichero en el 
  servidor. Actualizando sus metadatos en caso de que ya haya sido almacenado
  un fichero con su mismo contenido.
  \item \emph{Get}: Esta operación obtiene el contenido de un fichero específico
  de nuestro sistema a partir el identificador unívoco que lo representa.
  \item \emph{Remove}: Esta operación elimina un fichero del servidor a partir
  el identificador unívoco que lo representa.
  \item \emph{List}: Esta operación lista todos los ficheros del servidor que
  contengan todas las etiquetas dadas.
  \item \emph{Search}: Esta operación busca un texto libre sobre los datos
  de los ficheros almacenados en el servidores y devolviendo los ficheros que
  contienen los términos a buscar en alguno de sus metadatos (en esta búsqueda
  se excluye el contenido del fichero).
  \item \emph{Info}: Esta operación obtiene los elementos de un fichero
  específico almacenado en el servidor.
\end{itemize}

Cualquier aplicación que realice estas operaciones e interactúe con los
clientes de nuestro sistema puede ser considerada como servidor de nuestro
sistema.

Una especificación más profunda se encuentra en la \emph{Documentación del API}
de nuestro proyecto.
\end{document}